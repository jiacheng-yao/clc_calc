\documentclass{beamer} % "Beamer" is a word used in Germany to mean video projector. 

\usetheme{Berkeley} % Search online for beamer themes to find your favorite or use the Berkeley theme as in this file.

\usepackage{color} % It may be necessary to set PCTeX or whatever program you are using to output a .pdf instead of a .dvi file in order to see color on your screen.
\usepackage{graphicx} % This package is needed if you wish to include external image files.

\theoremstyle{definition} % See Lesson Three of the LaTeX Manual for more on this kind of "proclamation."
\newtheorem*{dfn}{Customer Lifetime Value for Customer $k$}               

\title{Customer Lifetime Value on Foodora and Foodpanda Datasets}
\author{Jiacheng Yao} 
\institute{Mathemads, Rocket Internet SE}
%\date{January 6, 2012} 
% Remove the % from the previous line and change the date if you want a particular date to be displayed; otherwise, today's date is displayed by default.

\AtBeginSection[]  % The commands within the following {} will be executed at the start of each section.
{
\begin{frame} % Within each "frame" there will be one or more "slides."  
\frametitle{Presentation Outline} % This is the title of the outline.
\tableofcontents[currentsection]  % This will display the table of contents and highlight the current section.
\end{frame}
} % Do not include the preceding set of commands if you prefer not to have a recurring outline displayed during your presentation.

\begin{document}

\begin{frame} 
\titlepage
\end{frame}

\section{Definition} % Since this is the start of a new section, our recurring outline will appear here.

\begin{frame} 
\frametitle{Definition: Customer Lifetime Value (CLV)}

\begin{dfn} % We created the proclamation dfn near the start of the document.
$CLV^{k} = \displaystyle\sum_{i=0}^{T} \cfrac{Profit_t^{k}}{(1+i_t)^{t}}$ 
\end{dfn}

\pause $i_t$ = Interest rate $i$ at Timepoint $t$ \\
\pause $T$ = Predicted lifetime period for customer $k$

\end{frame}

\begin{frame}
\frametitle{Approaches}

\begin{itemize}
\item Parametric Models
\begin{itemize}
\pause \item Pareto/NBD (Schmittlein et al., 1987)
\pause \item BG/NBD (Fader et al., 2005)
\pause \item etc 
\end{itemize}
\item Non-parametric Models
\end{itemize}

\end{frame}

\section{Implementation on the Datasets}

\begin{frame}
\frametitle{Cohort Analysis (Foodora)}

\begin{center}
            \includegraphics[width=\textwidth,height=0.8\textheight,keepaspectratio]{user_retention_full_fd.pdf}
        \end{center}
\end{frame}

\begin{frame}
\frametitle{Cohort Analysis (Foodora)}

\begin{center}
            \includegraphics[width=\textwidth,height=0.8\textheight,keepaspectratio]{user_retention_heatmap_full_fd.pdf}
        \end{center}
\end{frame}

\begin{frame}
\frametitle{Cohort Analysis (Foodpanda)}

\begin{center}
            \includegraphics[width=\textwidth,height=0.8\textheight,keepaspectratio]{user_retention_sg.pdf}
        \end{center}
\end{frame}

\begin{frame}
\frametitle{Cohort Analysis (Foodpanda)}

\begin{center}
            \includegraphics[width=\textwidth,height=0.8\textheight,keepaspectratio]{user_retention_heatmap_sg.pdf}
        \end{center}
\end{frame}

\begin{frame}
\frametitle{Modeling with BG/NBD (Foodora)}
\begin{center}
            \includegraphics[width=\textwidth,height=0.8\textheight,keepaspectratio]{f_r_matrix_fd.pdf}
        \end{center}
\end{frame}

\begin{frame}
\frametitle{Modeling with BG/NBD (Foodora)}
\begin{center}
            \includegraphics[width=\textwidth,height=0.8\textheight,keepaspectratio]{pr_alive_matrix_fd.pdf}
        \end{center}
\end{frame}

\begin{frame}
\frametitle{Modeling with BG/NBD (Foodora)}
\begin{center}
            \includegraphics[width=\textwidth,height=0.8\textheight,keepaspectratio]{period_transactions_fd.pdf}
        \end{center}
\end{frame}

\begin{frame}
\frametitle{Modeling with BG/NBD (Foodora)}
\begin{center}
            \includegraphics[width=\textwidth,height=0.8\textheight,keepaspectratio]{cal_vs_holdout_fd.pdf}
        \end{center}
\end{frame}

\begin{frame}
\frametitle{Modeling with BG/NBD (Foodpanda)}
\begin{center}
            \includegraphics[width=\textwidth,height=0.8\textheight,keepaspectratio]{f_r_matrix_sg_customers_recent.pdf}
        \end{center}
\end{frame}

\begin{frame}
\frametitle{Modeling with BG/NBD (Foodpanda)}
\begin{center}
            \includegraphics[width=\textwidth,height=0.8\textheight,keepaspectratio]{pr_alive_matrix_sg_customers_recent.pdf}
        \end{center}
\end{frame}

\begin{frame}
\frametitle{Modeling with BG/NBD (Foodpanda)}
\begin{center}
            \includegraphics[width=\textwidth,height=0.8\textheight,keepaspectratio]{period_transactions_sg_customers_recent.pdf}
        \end{center}
\end{frame}

\begin{frame}
\frametitle{Modeling with BG/NBD (Foodpanda)}
\begin{center}
            \includegraphics[width=\textwidth,height=0.8\textheight,keepaspectratio]{cal_vs_holdout_sg_customers_recent.pdf}
        \end{center}
\end{frame}

\begin{frame}
\frametitle{Modeling with BG/NBD + GammaGamma (Foodora)}
\begin{center}
            \begin{table}
\begin{tabular}{l | c | c | c }
Cohort Group & MAE & MAE/AVG & $R^2$ \\
\hline \hline
2015-06 & 17.558 & 0.605 & 0.648\\ 
2015-07 & 10.418 & 0.700 & 0.614\\
2015-08 & 9.092 & 0.747 & 0.506\\
2015-09 & 9.475 & 0.769 & 0.474\\ 
2015-10 & 9.291 & 0.780 & 0.563\\ 
2015-11 & 8.482 & 0.791 & 0.531\\
2015-12 & 8.099 & 0.838 & 0.475\\
2016-01 & 8.468 & 0.832 & 0.526\\ 
2016-02 & 8.963 & 0.858 & 0.433\\ 
2016-03 & 9.311 & 0.858 & 0.424\\ 
\end{tabular}
\caption{MAE etc. for the Cohort Groups}
\end{table}
        \end{center}
\end{frame}

\begin{frame}
\frametitle{Churn Rate Prediction with ModBetaGeo - Accuracy Evaluation (Foodora)}
\begin{center}
            \begin{table}
\begin{tabular}{l | c | c }
 & Predicted: churn & Predicted: alive\\
\hline \hline
Actual: churn & 15917 & 6360\\ 
Actual: alive & 4886 & 10138
\end{tabular}
\caption{Confusion Matrix for the Churn Rate Prediction}
\end{table}
        \end{center}
\newtheorem{rmk}{Remark}        
       
\begin{rmk}
	$ACC=0.698$ \\    
    $F_1=0.643$
\end{rmk}
\end{frame}

\begin{frame}
\frametitle{Churn Rate Prediction with RF - Accuracy Evaluation (Foodora)}
\begin{center}
            \begin{table}
\begin{tabular}{l | c | c }
 & Predicted: churn & Predicted: alive\\
\hline \hline
Actual: churn & 19097 & 3180\\ 
Actual: alive & 6802 & 8222
\end{tabular}
\caption{Confusion Matrix for the Churn Rate Prediction with RF}
\end{table}
        \end{center}      
       
\begin{rmk}
	$ACC=0.732$ \\    
    $F_1=0.622$
\end{rmk}
\end{frame}

\begin{frame}
\frametitle{Churn Rate Prediction with XGBoost - Accuracy Evaluation (Foodora)}
\begin{center}
            \begin{table}
\begin{tabular}{l | c | c }
 & Predicted: churn & Predicted: alive\\
\hline \hline
Actual: churn & 19564 & 2713\\ 
Actual: alive & 6991 & 8033
\end{tabular}
\caption{Confusion Matrix for the Churn Rate Prediction with XGBoost}
\end{table}
        \end{center}      
       
\begin{rmk}
	$ACC=0.740$ \\    
    $F_1=0.623$
\end{rmk}
\end{frame}

\begin{frame}
\frametitle{Modeling with BG/NBD + GammaGamma (Foodpanda)}
\begin{center}
            \begin{table}
\begin{tabular}{l | c | c | c }
Cohort Group & MAE & MAE/AVG & $R^2$ \\
\hline \hline
2015-06 & 6.089 & 0.754 & 0.497\\ 
2015-07 & 4.083 & 0.839 & 0.364\\
2015-08 & 3.693 & 0.862 & 0.332\\
2015-09 & 3.598 & 0.869 & 0.379\\ 
2015-10 & 4.643 & 0.881 & 0.195\\ 
2015-11 & 5.514 & 0.861 & 0.357\\
2015-12 & 5.108 & 0.875 & 0.363\\
2016-01 & 5.819 & 0.875 & 0.337\\ 
2016-02 & 5.978 & 0.901 & 0.299\\ 
2016-03 & 7.107 & 0.886 & 0.508\\ 
\end{tabular}
\caption{MAE etc. for the Cohort Groups}
\end{table}
        \end{center}
\end{frame}

\begin{frame}
\frametitle{Churn Rate Prediction with ModBetaGeo - Accuracy Evaluation (Foodpanda)}
\begin{center}
            \begin{table}
\begin{tabular}{l | c | c }
 & Predicted: churn & Predicted: alive\\
\hline \hline
Actual: churn & 61927 & 3353\\ 
Actual: alive & 11247 & 7424
\end{tabular}
\caption{Confusion Matrix for the Churn Rate Prediction}
\end{table}
        \end{center}     
       
\begin{rmk}
	$ACC=0.826$ \\
    $F_1=0.504$ 
\end{rmk}
\end{frame}

\begin{frame}
\frametitle{Churn Rate Prediction with RF - Accuracy Evaluation (Foodpanda)}
\begin{center}
            \begin{table}
\begin{tabular}{l | c | c }
 & Predicted: churn & Predicted: alive\\
\hline \hline
Actual: churn & 62245 & 3035\\ 
Actual: alive & 10813 & 7858
\end{tabular}
\caption{Confusion Matrix for the Churn Rate Prediction with RF}
\end{table}
        \end{center}      
       
\begin{rmk}
	$ACC=0.826$ \\    
    $F_1=0.532$
\end{rmk}
\end{frame}

\begin{frame}
\frametitle{Churn Rate Prediction with XGBoost - Accuracy Evaluation (Foodpanda)}
\begin{center}
            \begin{table}
\begin{tabular}{l | c | c }
 & Predicted: churn & Predicted: alive\\
\hline \hline
Actual: churn & 19564 & 2713\\ 
Actual: alive & 6991 & 8033
\end{tabular}
\caption{Confusion Matrix for the Churn Rate Prediction with XGBoost}
\end{table}
        \end{center}      
       
\begin{rmk}
	$ACC=0.740$ \\    
    $F_1=0.623$
\end{rmk}
\end{frame}

\begin{frame}
\frametitle{Transaction Count Prediction - Comparison with ZodiacMetrics (Foodpanda)}
\begin{center}
            \begin{table}
\begin{tabular}{l | c| c | c | c}
 & Cohort & MAE & MAE/AVG & $R^2$\\
\hline \hline
M & 2015-Q1 & 0.813 & 1.131 & 0.371\\ 
Z & & 0.830 & 1.156 & 0.266\\
\hline \hline
M & 2015-Q2 & 0.836 & 0.984 & 0.374\\ 
Z & & 0.817 & 0.962 & 0.307\\
\hline \hline
M & 2013-Q1 & 1.136 & 0.873 & 0.415\\ 
Z & & 1.200 & 0.922 & 0.293
\end{tabular}
\caption{Comparison between ModBetaGeo (M) and ZodiacMetrics (Z)}
\end{table}
        \end{center}      
       
\end{frame}

\begin{frame}
\frametitle{Future Total Sales Prediction - Comparison with ZodiacMetrics (Foodpanda)}
\begin{center}
            \begin{table}
\begin{tabular}{l | c| c | c | c}
 & Cohort & MAE & MAE/AVG & $R^2$\\
\hline \hline
M & 2015-Q1 & 0.307 & 0.909 & 0.315\\ 
Z & & 0.343 & 1.015 & 0.281\\
\hline \hline
M & 2015-Q2 & 0.362 & 0.892 & 0.306\\ 
Z & & 0.382 & 0.941 & 0.306\\
\hline \hline
M & 2013-Q1 & 0.419 & 0.828 & 0.374\\ 
Z & & 0.464 & 0.917 & 0.311
\end{tabular}
\caption{Comparison between ModBetaGeo (M) and ZodiacMetrics (Z)}
\end{table}
        \end{center}      
       
\end{frame}


\section{Further Work} % Since this is the start of a new section, our recurring outline will appear here.
\begin{frame}
\frametitle{Future Work}
\begin{itemize}
\item Model with more methodologies and make detailed comparisons
\item Assess the impact of the calibration/holdout set ratio on model accuracy
\item Understand better the statistical and behavioral differences between the two datasets
\item etc
\end{itemize}
\end{frame}

\begin{frame}
Questions?
\end{frame}

% Some of the ideas for the functions presented in this document came from posts to the ap-calculus group at the Math Forum at http://mathforum.org/kb/forum.jspa?forumID=63.

\end{document}

